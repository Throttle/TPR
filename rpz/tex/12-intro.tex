\Introduction
Настоящее техническое задание разработано в рамках учебной программы по курсу <<Технология программирования>> на программное изделие <<Распределенная система обнаружения и фильтрации спама в протоколах POP3, IMAP, SMTP>>. Техническое задание выполняется в соответствии со стандартом ГОСТ 34.602-89 <<Техническое задание на создание автоматизированной системы>>.

\section{Наименование предприятия разработчика и заказчика системы}
Разработчиком системы является Сулимов Александр Сергеевич, студент группы ИУ7-104 кафедры ИУ-7 <<Программное обеспечение вычислительной техники и информационные технологии>> пятого курса очной формы обучения МГТУ им. Баумана.

Заказчиком системы является кафедра ИУ-7 <<Программное обеспечение вычислдительной техники и информационные технологии>> МГТУ им. Баумана (далее Заказчик).

\section{Основания для разработки}
Разработка ведется в рамках выполнения лабораторных работ по курсу <<Технология программирования>> и курсового проекта по курсу <<Распределенные системы обработки информации>> в соответствии с учебным планом МГТУ им. Баумана на 10-й семестр для факультета ИУ кафедры ИУ-7. При выполнении работы учитываются указания, описанные в методических пособиях \cite{metodRomanova}, \cite{metodKKrishenko}.

\section{Сроки выполнения работ по созданию системы}
В соответствии с требованиями Заказчика установлены следующие сроки (указываются номера недель, начиная с 6 февраля 2012 года):

\begin{enumerate}
	\item начало выполнения работ --- 3 неделя (20.02.2012 г.);
	\item конец выполнения работ --- 15 неделя (14.05.2012 г.).
\end{enumerate}

\section{Порядок оформления и предъявления результатов работы по созданию системы}
В \cite{metodKKrishenko} определены следующие требования к оформлению и защите проекта:

\begin{enumerate}
	\item в процессе разработки системы необходимо использовать выделенный Заказчиком репозиторий системы контроля версий;
	\item ответственное лицо со стороны Заказчика ведет контроль выполнения работы по репозиторию, учитывая количество коммитов, определяет процент выполнения проекта;
	\item результаты работ представляются в форме защиты проекта, на которой необходимо предъявить:
		\begin{enumerate}
			\item оформленную, прошитую и подписанную руководителем проекта пояснительную записку;
			\item презентацию в электронном виде в формате PDF;
			\item устаноленное и работающее программное обеспечение с исходными кодами, хранимыми в репозитории;
		\end{enumerate}
	\item при защите программное обеспечение должно быть установленно на компьютерах Заказчика;
	\item для защиты проекта создается комиссия из двух или более ответственных лиц со стороны Заказчика.
\end{enumerate}


\section{Актуальность разрабатываемой системы}
Представленные системы, позволяющие фильтровать почту он нежелательных сообщений, основываются на спам-фильтрах, которые имеют следующие недостатки:

\begin{enumerate}
\item{необходимо обучение;}
\item{история обучения локальная и относится к конкретному пользователю.}
\end{enumerate}

Настоящая разработка должна обеспечить передачу информации между пользователями о признаках обнаруженного спама. Распределенная система обнаружения спама позволит сократить время на обучение спам-фильтров для отдельных пользователей, тем самым повысит удобство использования почтовых клиентов.



\section{Краткое описание предметной области}
За последние десять лет сфера применения спама расширилась, а объем доставки - вырос значительно. Первое время спам рассылался напрямую на единичные адреса пользователей, и его было легко блокировать. Со временем появились высокоскоростные интернет-каналы, которые дали быструю и дешевую возможность массово рассылать спам-сообщения. Модемы пользователей не оснащались средствами защиты от несанкционированного доступа и могли использоваться злоумышленниками из любой точки планеты. Другими словами, модемы ничего не подозревающих пользователей рассылали огромные объемы спама.

Так продолжалось, пока производители аппаратного обеспечения не научились оснащать оборудование средствами защиты от спама, а спам-фильтры не стали более эффективными. Однако спам тоже эволюционировал: усовершенствовались не только способы рассылки, но и приемы, помогающие злоумышленникам обойти спам-фильтры. 

В ходе анализа предметной области были рассмотрены наиболее популярные спам-фильтры и другие механизмы, применяющиеся для фильтрации потока электронной почты от спама.


\section{Существующие аналоги}
В рамках настоящей работы был произведен анализ рынка программных продуктов, позволяющих фильтровать электронную почту от спама. Было выявлено, что существует множество решений позволяющих осуществлять определение нежелательной почты для конкретного пользователя. 

Существует два класса подобных продуктов:
\begin{enumerate}
	\item спам-фильтры на почтовых клиентах;
	\item спам-фильтры на почтовых серверах. 
\end{enumerate}

Спам-фильтры почтовых клиентов основаны на обработке входящих сообщений согласно заданным пользовательским правилам.  Почтовые сервера используют так называемые черные и серые списки доверенных адресов входящей почты.

Таким образом, распределенных систем, позволяющих осуществлять фильтрацию спама обнаружено не было.


\section{Назначение и цели создания системы}
Назначение разработки --- удовлетворить потребности клиентов (пользователей электронной почтой) в получении электронных писем, не относящихся к спаму.

Цели системы:
\begin{enumerate}
	\item информировать пользователя о том, что пришедшее сообщение относится к спаму;
	\item предоставлять пользователю возможность самостоятельно формировать спам-фильтры;
	\item обеспечить доставку писем, не относящихся к спаму.
\end{enumerate}

Таким образом, субъектами разрабытваемой РСОИ являются:

\begin{enumerate}
\item{система <<А>>}, почтовый клиент (MUA);
\item{система <<Б>>}, почтовый сервер (MTA);
\item{система <<C>>}, анти-спам сервер. 
\end{enumerate}

Перечисленные системы в дальнейшем будут также называться субъектами РСОИ или системами-участниками.

На рисунке \ref{fig:structure} представлена структура предметной области и ее составные части.

\begin{figure}
  \centering
  \includegraphics[width=\textwidth]{inc/dia/rsoi-structure}
  \caption{Структура предметной области}
  \label{fig:structure}
\end{figure}


\subsection{Система <<А>>}
Целью систем данного типа является предоставление удобного пользовательского интерфейса для работы с электронной почтой. С точки зрения почтовой системы представляют собой MUA --- программное обеспечение, устанавливаемое на компьютере пользователя и предназначенное для получения, написания, отправки и хранения сообщений электронной почты одного или нескольких пользователей (в случае, например, нескольких учётных записей на одном компьютере) или нескольких учётных записей одного пользователя. 

Системы <<А>> бывают двух типов в зависимости от протокола получения входящих сообщений:
\begin{enumerate}
\item{POP3;}
\item{IMAP.}
\end{enumerate}

Протокол POP3 подразумевает передачу входящих сообщений почтовым сервером и сохранение электронных писем на локальном компьютере пользователя. При использовании POP3 клиент подключается к серверу только на промежуток времени, необходимый для загрузки новых сообщений. При использовании IMAP соединение не разрывается, пока пользовательский интерфейс активен, а сообщения загружаются только по требованию клиента. Это позволяет уменьшить время отклика для пользователей, в чьих ящиках имеется много сообщений большого объёма.
Протокол POP требует, чтоб текущий клиент был единственным подключенным к ящику. IMAP позволяет одновременный доступ нескольких клиентов к ящику и предоставляет клиенту возможность отслеживать изменения, вносимые другими клиентами, подключенными одновременно с ним.


\subsection{Система <<Б>>}
Системы аккумулируют информацию необходимую для фильтрации электронной почты пользователя от спама. Целевые пользователи систем <<А>> должны зарегистрироваться в одной из систем типа <<Б>>. Такие системы получают информацию от пользователей относительно того, являются ли те или иные сообщения спамом или нет. В зависимости от полученных данных от пользователей в системах типа <<Б>> аккумулируется база данных, содержащая список адресов, с которых рассылаются спам-сообщения.

При получении почты системой <<А>> посылается запрос связанной с ней системой <<Б>>, которая должна классифицировать входящее сообщение на наличе спама.

Системы типа <<Б>> взаимодействуют между собой, таким образом пользователи, входящие в состав РСОИ, обладают суммарной информацией о всех спам-рассылках в рамках текущей сети. Так как с каждым пользователем системы <<А>>  в определенный момент времени связан лишь один антиспам сервер, то существует возможность составлять персональные спам-списки для каждого пользователя. 


\subsection{Система <<В>>}
 Системы данного типа отвечает за отправку почты и представляют собой почтовые сервера, которые обычно выполняют роль MTA и MDA. Некоторые почтовые сервера (программы) выполняют роль как MTA, так и MDA, некоторые подразумевают разделение на два независимых сервера: сервер-MTA и сервер-MDA (при этом, если для доступа к почтовому ящику используются различные протоколы — например, POP3 и IMAP, — то MDA в свою очередь может быть реализован либо как единое приложение, либо как набор приложений, каждое из которых отвечает за отдельный протокол).

 \section{Требования к системе}
 \subsection{Требование к системе вцелом}

 Разрабатываемое ПО должно удовлетворять следующим требованиям:

 \begin{enumerate}
 	\item все системы в РСОИ могут быть в одном или нескольких экземплярах, включение новой системы в РСОИ не должно приводить к нарушению работы других субъектов;
 	\item программное обеспечение для систем каждого типа должно 	поддерживать функционирование системы в режиме:
 		\begin{enumerate}
 			\item системы <<А>> --- периодическая работа в зависимости от желаний пользователя, в том числе поддерживать режим 24/7/365;
 			\item системы <<Б>> --- режим 24/7/365;
 			\item системы <<В>> --- режим 24/7/365.
 		\end{enumerate}
 	\item системы должны быть устойчивы к отключениям питания и другим техническим сбоям, которые могут привести к нарушению нормального фуннкционирования систем;
 	\item выход из строя одного субъекта РСОИ не должен приводить к сбою в работе других систем;
 	\item для корректного взаимодействия систем в РСОИ должен быть разработан протокол, однозначно определяющий содержание всех запросов и возможные сценарии совместной работы систем с указанием возможных состояний каждой из систем и состояний заявок в них.
 \end{enumerate}

 \subsection{Требования к функциональным характеристикам}
 К разрабатываему ПО выдвигаются следующие функциональные требования:
 \begin{enumerate}
 	\item время отклика на запрос пользователя не должно превышать 3 секунд;
 	\item время ожидания запросов для различных систем должно настраиваться с помощью конфигурационных файлов отдельно;
 	\item системы типа <<Б>> должны функционировать и поддерживать указанное время отклика для 50 одновременно подключенных клиентов-систем типа <<А>>;
 	\item указанные требования к временным задержкам должны соблюдаться для систем всех типов при одноверменной работе с 5-10 системами партнерами.
 \end{enumerate}

 \subsection{Требования по реализации}
 При разработке РСОИ участвующие системы реализуются в виде независимых программных продуктов, использующих общий протокол прикладного уровня для взаимодействия друг с другом. 

 К каждому из программных продуктов предъявляются следующие тербования:
 \begin{enumerate}
 	\item разрабатываемое ПО должно предоставлять удобный пользовательский интерфейс для работы администраторов и пользователей;
 	\item интерфейс систем должен быть реализован как WEB-интерфейс;
 	\item каждая заявка должна обладать уникальным идентификатором в рамках одного субъекта РСОИ;
 	\item в качестве транспортного протокола системы должны использовать почтовые протоколы SMTP для отправки POP3/IMAP для получения сообщений и протоколы удаленного вызова процедур XML-RPC;
 	\item в качестве языка разметки могут быть использованы языки XML, либо JSON;
 	\item для хранения данных о пользователях, сообщениях, спаме необходимо использовать СУБД Postgres. Непосредственный доступ к базе данных одной системы должен быть закрыт для систем партнеров и клиентов.
 \end{enumerate}


\section{Функциональные требования к системе}
\subsection{Функциональные требования к системам типа <<А>>}
Система <<А>> должна предоставлять следующие функции:
\begin{enumerate}
	\item аутентификация пользователей;
	\item просмотр входящих сообщений пользователями;
	\item регистрация анти-спам сервера через панель настроек;
	\item оповещение связанного анти-спам сервера о наличии спам сообщений;
	\item просмотр сообщений, которые являются спамом по данным РСОИ;
	\item добавление адресатов в доверенный список, после чего сообщения от данных пользователей не классифицируются как спам.
\end{enumerate}



\subsection{Функциональные требования к системам типа <<Б>>}

 
\subsection{Функциональные требования к системам типа <<В>>}

\section{Входные параметры}
\subsection{Входные параметры систем типа <<А>>}
\subsection{Входные параметры систем типа <<Б>>}
\subsection{Входные параметры систем типа <<В>>}


\section{Требования к составу и параметрам технических средств}

\section{Сценарии функционирования системы}


\section{Состав и содержание работ по созданию системы}
В таблице \ref{tab:sostav} представлен состав работ по созданию РСОИ <<Распределенная система обнаружения и фильтрации спама в протоколах POP3, IMAP, SMTP>>. Перечень работ соответствует требованиям Заказчика.

\begin{table}[ht]
  \caption{Перечень работ по созданию РСОИ}
  \begin{tabular}{}
  \hline
  Выполняемая работа & Срок выполнения\\
  \hline
  Исследование объектов автоматизации, сбор сведений о существующих аналогах & 1 неделя \\
  \hline
  Разработка технического задания & 4 неделя \\
  \hline
  Разработка пилотного проекта по выбранному варианту РСОИ & 6 неделя \\
  \hline
  Разработка окончательных решений по выбранным структурам, разработка конечных вариантов процедур и заглуушек систем-партнеров & 8 неделя \\
  \hline
  Разработка пользовательского интерфейса & 12 неделя \\ 
  \hline
   Создание документации & 12 неделя \\
  \hline
  Отладка проекта, подготовка к защите & 14 неделя \\
  \hline
  Защита проекта & 15 неделя\\
  \hline
  \end{tabular}
  \label{tab:sostav}
\end{table}


\section{Порядок контроля и приемки системы}
При разработке системы необходимо произвести следующие испытания:
\begin{enumerate}
	\item тестирование логики работы систем всех типов <<А>>, <<Б>>, <<В>>: при тестировании системы одного типа системы другого типа представляются в виде заглушек, работающих по строго заданным сценариям;
	\item тестирование нормальной совместной работы систем <<А>>, <<Б>>, <<В>>: проверяются различные сценарии работы при корректных входных данных, предполагаемые результаты работы сравниваются с реально полученными;
	\item испытание РСОИ на отказоустойчивость: имитирование таких событий как отключение питания, выход из строя других систем в составе РСОИ, поступление запросов с ошибочной или противоречивой информацией, поступление запросов в неверном формате и порядке.
\end{enumerate}


\section{Требования к документации}
Разработка, установка и внедрение системы должны сопровождаться следующими документами:

\begin{enumerate}
	\item руководство по установке, настройке систем всех типов при развертывании РСОИ;
	\item руководство по использованию систем <<А>> для пользователей;
	\item руководство по использованию систем <<Б>>, <<В>> для администраторов.
\end{enumerate}


























