\Introduction

Почтовая программа (клиент электронной почты, почтовый клиент, мейл-клиент, мейлер) — программное обеспечение, устанавливаемое на компьютере пользователя и предназначенное для получения, написания, отправки и хранения сообщений электронной почты одного или нескольких пользователей (в случае, например, нескольких учётных записей на одном компьютере) или нескольких учётных записей одного пользователя. Их предназначением является предоставление удобного пользовательского интерфейса для работы с электронной почтой. В рамках целой почтовой системы почтовые клиенты являются лишь интерфейсом пользователя. Когда пользователь отправляет письмо через почтовый клиент, это не означает, что данное приложение непосредственно 



Целью работы является создание всякой всячины. Для достижения поставленной цели необходимо решить следующие задачи:

\begin{itemize}
\item проанализировать существующую всячину;
\item спроектировать свою, новую всячину;
\item изготовить всякую всячину;
\item проверить её работоспособность.
\end{itemize}

Вот так-то. А этот абзац вставлен для визуальной оценки отступа от перечня до следующего абзаца.