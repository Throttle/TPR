\Introduction
Настоящее техническое задание разработано в рамках учебной программы по курсу <<Технология программирования>> на программное изделие <<Распределенная система обнаружения и фильтрации спама в протоколах POP3, IMAP, SMTP>>.


\section{Краткое описание предметной области}
За последние десять лет сфера применения спама расширилась, а объем доставки - вырос значительно. Первое время спам рассылался напрямую на единичные адреса пользователей, и его было легко блокировать. Со временем появились высокоскоростные интернет-каналы, которые дали быструю и дешевую возможность массово рассылать спам-сообщения. Модемы пользователей не оснащались средствами защиты от несанкционированного доступа и могли использоваться злоумышленниками из любой точки планеты. Другими словами, модемы ничего не подозревающих пользователей рассылали огромные объемы спама.

Так продолжалось, пока производители аппаратного обеспечения не научились оснащать оборудование средствами защиты от спама, а спам-фильтры не стали более эффективными. Однако спам тоже эволюционировал: усовершенствовались не только способы рассылки, но и приемы, помогающие злоумышленникам обойти спам-фильтры. 

В ходе анализа предметной области были рассмотрены наиболее популярные спам-фильтры и другие механизмы, применяющиеся для фильтрации потока электронной почты от спама.



\section{Существующие аналоги}
В рамках настоящей работы был произведен анализ рынка программных продуктов, позволяющих фильтровать электронную почту от спама. Было выявлено, что существует множество решений позволяющих осуществлять определение нежелательной почты для конкретного пользователя. 

Существует два класса подобных продуктов:
\begin{enumerate}
	\item спам-фильтры на почтовых клиентах;
	\item спам-фильтры на почтовых серверах. 
\end{enumerate}

Спам-фильтры почтовых клиентов основаны на обработке входящих сообщений согласно заданным пользовательским правилам.  Почтовые сервера используют так называемые черные и серые списки доверенных адресов входящей почты.

Таким образом, распределенных систем, позволяющих осуществлять фильтрацию спама обнаружено не было.

\section{Описание системы}
Назначением разрабатываемой системы является осуществление фильтрации спама, которая основывается на информации от многих пользователей, которые имебт почтовые ящики на различных почтовых серверах.


\section{Основания для разработки}
Разработка ведется в рамках выполнения лабораторных работ по курсу <<Технология программирования>> и курсового проекта по курсу <<Распределенные системы обработки информации>> на основании учебного плана МГТУ им. Баумана на 12-й семестр для факультета ИУ.

\section{Назначение разработки}