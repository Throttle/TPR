\chapter{Аналитический раздел}
\label{cha:analysis}
%
% % В начале раздела  можно напомнить его цель
%
В данном разделе анализируется и классифицируется существующая всячина и пути создания новой всячины. А вот отступ справа в 1 см.~--- это хоть и по ГОСТ, но ведь диагноз же...

\section{Структура почтовой системы}

Почтовая система в общем случае может быть представлена как совокупность следующих элементов:
\begin{itemize}
  \item MUA --- (англ. Mail User Agent) почтовый агент пользователя, почтовый клиент);
  \item MTA --- (англ. Mail Transfer Agent) агент пересылки почты;
  \item MDA --- (англ. Mail Delivery Agent) агент доставки почты.
\end{itemize}

MUA представляет собой программное обеспечение, устанавливаемое на компьютере пользователя и предназначенное для получения, написания, отправки и хранения сообщений электронной почты одного или нескольких пользователей (в случае, например, нескольких учётных записей на одном компьютере) или нескольких учётных записей одного пользователя. Предназначением является предоставление удобного пользовательского интерфейса для работы с электронной почтой.

MTA отвечает за отправку почты. MDA отвечает за доставку почты конечному пользователю. Почтовые сервера обычно выполняют роль MTA и MDA. Некоторые почтовые сервера (программы) выполняют роль как MTA, так и MDA, некоторые подразумевают разделение на два независимых сервера: сервер-MTA и сервер-MDA (при этом, если для доступа к почтовому ящику используются различные протоколы — например, POP3 и IMAP, — то MDA в свою очередь может быть реализован либо как единое приложение, либо как набор приложений, каждое из которых отвечает за отдельный протокол).

Только один MTA может функционировать на одной рабочей станции, так как исключительно одно приложение может быть назначено для получения входящих сообщений от других рабочих станций. Как правило, это привелигерованная программа, которая прослушивает входящие TCP/IP соединения по SMTP-порту и имеет возможность сохранять данные в пользовательчкие почтовые ящики.

MTA способен принимать множество сообщений одновременно. Если в связи с непредвиденными обстоятельствами MTA не может доставить сообщение конечному пользователю, то посылается сообщение с причиной неудачной отправки. MTA хранит все сообщения, которые не могут быть своевременно отправлены конечному пользователю. Через определенные промежутки времени MTA инициирует повтоную отправку подобных сообщений. Чаще всего невозможность доставки почты связана с проблемами сетевого соединения и с отключением целевой рабочей станции.

С точки зрения MTA существуют два типа источников входящих сообщений: локальные процессы и другие рабочие станции. Также можно выделить три типа адресатов: локальные файлы, локальные процессы и другие рабочие станции.

Выделение агентов MTA и MUA означает, что они могут функционировать на различных рабочих станциях.

В верхней части рисунка \ref{fig:} MUA, MTA и дисковое хранилище являются частью единой системы, которая выделена штриховой линией. Пользователи получают доступ к системе пктем аутентификации и авторизации с помощью ввода логина и соответсчтвующего пароля. MUA запускается с помощью пользовательской команды, как процесс операционной системы, и когда иницирует передачу сообщения MTA для последующей отправки, начинает ввзаимодействие с другим процессом опрерационной системы. MUA и MTA взаимодействуют с авторизованным пользователем, поэтому MTA обычно проивзодит проверку того, что идентификационная информация пользователя включена в исходящее сообщение. Как описано в RFC 822, если в теле сообщения не указывается адресант From, то MTA обязан добавить запись Sender с указанием идентификационной информациии отправителя.

MTA хранит сообщения в так называемой области спулинга. Под областью спулинга будем понимать дисковое хранилище, которое использует MTA для временного хранения очереди сообщений перед их отправкой.

Сообщения, которые предназначены другим удаленным рабочим станциям передаются по сети Интернет другим MTA с использованием протокола Simple Mail Transfer Protocol (SMTP). В том случае когда сервер-отправитель и сервер-получатель оба напрямую подключены к сети Интернет, сообщение может быть доставлено напрямую от отправитля к получателю. Однако иногда сообщению приходится преодолевать маршруты через промежуточные MTA. Большие организации часто организуют свои почтовые системы таким образом, что все входящие сообщения поступают на главный почтовый сервер. После этого сообщения пересылаются через другие серверы локальной сети. Когда сообщение доставляется получателю, MTA сохраняет его в почтовом ящике пользователя, который затем с помощью MUA получает доступ к своей почте.

Также промежуточные MTA используются в тех случаях, когда целевой почтовый сервер не доступен или сетевое соединение не может быть установлено. Преимущество такого подхода заключается в том, что сообщения аккумулируются на серверах максимально приближенных к целевому адресату и могут быть своевременно доставлены.

Нижняя часть рисунка иллюстрирует такую конфигурацию почтовой подсистемы получателя, когда MUA и MTA располагаются на различных рабочих станциях. Такая конфигурация позволяет разделить процессы получения и отправки писем на отдельные операции. В процессе чтения почты пользователь напрямую взаимодействует с MUA, который использует протоколы POP3 (RFC 1939) или IMAP (RFC 2060) для получения доступа к почтовым ящикам пользователя и удаленным папкам на сервере. Для того чтобы осуществить подобные операции, пользователь должен быть авторизован в почтовой системе. Однако, протоколы POP3 и IMAP не содержат средств для пересылки сообщений. MUA подобного типа используют протокол SMTP для отправки сообщения MTA. Таким образом протокол SMTP, который изначально описывал передачу сообщений между MTA, в настоящее вреся используется также для транзита сообщений от MUA к MTA. Такое использование приводит к ряду проблем:

\begin{itemize}
\item MTA не может различить сообщения, поступающие от других MTA, от сообщений, предоставляемых MUA. 
\item Пользователь, отправляющий почту, не является установленным, то есть прошедшим авторизацию. В связи с этим MTA не предоставляется возможным определить является ли домен отправителя реально существующим.
\item MUA может использовать различные сервера для отправки почты. Также специализированные MUA могут отправлять сообщения адресату напрямую через сеть интернет. Поэтому возможна ситуация, когда злоумышленники пытаются отправить поток нежелательной почты на произвольные сервера для ретрасляции.
\end{itemize}

На данный момент разработаны протоколы, которые способны решать описанные проблемы. Однако широкого применения они не получили.


\section{Типы MTA}
В простейшем случае отдельные рабочие станции или сервера в небольших офисах и домах, которые работают с несколькими почтовыми ящиками в одном домене, получают входящие сообщения от одного провайдера интернет услуг и передают сообщения провайдеру для последующей доставки адресату. Среди хостов, имеющих постоянное подключение к сети интернет, также имеются хосты, которые периодически подключаются к интернету для получения входящей почты  с сервера и отправки исходящей почты. 

Рабочие станции, которые имеют постоянное подключение к сети интернет, могут не посылать сообщения через один и тот же сервер. 


\section{Стандарты, регламентирующие передачу сообщений по сети Интернет}
Передача сообщений по сети интернет регламентируется согласно стандарту RFC 822, в котором определяется формат передаваемых сообщений. Протокол SMTP обмена сообщениями между хостами описывается в стандартах RFC 821, RFC 1123, а также в некоторых других стандартах, определяющих расширения протокола SMTP.



% Обратите внимание, что включается не ../dia/..., а inc/dia/...
% В Makefile есть соответствующее правило для inc/dia/*.pdf, которое
% берет исходные файлы из ../dia в этом случае.

\begin{figure}
  \centering
  \includegraphics[width=\textwidth]{inc/dia/rpz-idef0}
  \caption{Рисунок}
  \label{fig:fig01}
\end{figure}

В \cite{Pup09} указано, что...

Кстати, про картинки. Во-первых, для фигур следует использовать \texttt{[ht]}. Если и после этого картинки вставляются <<не по ГОСТ>>, т.е. слишком далеко от места ссылки,~--- значит у вас в РПЗ \textbf{слишком мало текста}! Хотя и ужасный параметр \texttt{!ht} у окружения \texttt{figure} тоже никто не отменял, только при его использовании документ получается страшный, как в ворде, поэтому просьба так не делать по возможности.

\section{Существующие подходы к созданию всячины}

Известны следующие подходы...

\begin{enumerate}
\item Перечисление с номерами.
\item Номера первого уровня. Да, ГОСТ требует именно так~--- сначала буквы, на втором уровне~--- цифры.
Чуть ниже будет вариант <<нормальной>> нумерации и советы по её изменению.
Да, мне так нравится: на первом уровне выравнивание элементов как у обычных абзацев. Проверим теперь вложенные списки.
\begin{enumerate}
\item Номера второго уровня.
\item Номера второго уровня. Проверяем на длииииной-предлиииииииииинной строке, что получается.... Сойдёт.
\end{enumerate}
\item По мнению Лукьяненко, человеческий мозг старается подвести любую проблему к выбору
  из трех вариантов.
\item Четвёртый (и последний) элемент списка.
\end{enumerate}

Теперь мы покажем, как изменить нумерацию на «нормальную», если вам этого захочется. Пара команд в начале документа поможет нам.

\renewcommand{\labelenumi}{\arabic{enumi})}
\renewcommand{\labelenumii}{\asbuk{enumii})}

\begin{enumerate}
\item Изменим нумерацию на более привычную...
\item ... нарушим этим гост.
\begin{enumerate}
\item Но, пожалуй, так лучше.
\end{enumerate}
\end{enumerate}

В заключение покажем произвольные маркеры в списках. Для них нужен пакет \textbf{enumerate}.
\begin{enumerate}[1.]
\item Маркер с арабской цифрой и с точкой.
\item Маркер с арабской цифрой и с точкой.
\begin{enumerate}[I.]
\item Римская цифра с точкой.
\item Римская цифра с точкой.
\end{enumerate}
\end{enumerate}

В отчётах могут быть и таблицы~--- см. табл.~\ref{tab:tabular} и~\ref{tab:longtable}.
Небольшая таблица делается при помощи \Code{tabular} внутри \Code{table} (последний
полностью аналогичен \Code{figure}, но добавляет другую подпись).

\begin{table}[ht]
  \caption{Пример короткой таблицы с длинным названием на много длинных-длинных строк}
  \begin{tabular}{|r|c|c|c|l|}
  \hline
  Тело      & $F$ & $V$  & $E$ & $F+V-E-2$ \\
  \hline
  Тетраэдр  & 4   & 4    & 6   & 0         \\
  Куб       & 6   & 8    & 12  & 0         \\
  Октаэдр   & 8   & 6    & 12  & 0         \\
  Додекаэдр & 20  & 12   & 30  & 0         \\
  Икосаэдр  & 12  & 20   & 30  & 0         \\
  \hline
  Эйлер     & 666 & 9000 & 42  & $+\infty$ \\
  \hline
  \end{tabular}
  \label{tab:tabular}
\end{table}

Для больших таблиц следует использовать пакет \Code{longtable}, позволяющий создавать
таблицы на несколько страниц по ГОСТ.

Для того, чтобы длинный текст разбивался на много строк в пределах одной ячейки, надо в
качестве ее формата задавать \texttt{p} и указывать явно ширину: в мм/дюймах
(\texttt{110mm}), относительно ширины страницы (\texttt{0.22\textbackslash textwidth})
и~т.п.

Можно также использовать уменьшенный шрифт~--- но, пожалуйста, тогда уж во \textbf{всей}
таблице сразу.

\begin{center}
  \begin{longtable}{|p{0.40\textwidth}|c|p{0.30\textwidth}|}
    \caption{Пример длинной таблицы с длинным названием на много длинных-длинных строк}
    \label{tab:longtable}
    \\ \hline
    Вид шума & Громкость, дБ & Комментарий \\
    \hline \endfirsthead
    \subcaption{Продолжение таблицы~\ref{tab:longtable}}
    \\ \hline \endhead
    \hline \subcaption{Продолжение на след. стр.}
    \endfoot
    \hline \endlastfoot
    Порог слышимости             & 0     &                                                \\
    \hline
    Шепот в тихой библиотеке     & 30    &                                                \\
    Обычный разговор             & 60-70 &                                                \\
    Звонок телефона              & 80    & \small{Конечно, это было до эпохи мобильников} \\
    Уличный шум                  & 85    & \small{(внутри машины)}                        \\
    Гудок поезда                 & 90    &                                                \\
    Шум электрички               & 95    &                                                \\
    \hline
    Порог здоровой нормы         & 90-95 & \small{Длительное пребывание на более
    громком шуме может привести к ухудшению слуха}                                        \\
    \hline
    Мотоцикл                     & 100   &                                                \\
    Power Mower                  & 107   & \small{(модель бензокосилки)}                  \\
    Бензопила                    & 110   & \small{(Doom в целом вреден для здоровья)}     \\
    Рок-концерт                  & 115   &                                                \\
    \hline
    Порог боли                   & 125   & \small{feel the pain}                          \\
    \hline
    Клепальный молоток           & 125   & \small{(автор сам не знает, что это)}          \\
    \hline
    Порог опасности              & 140   & \small{Даже кратковременное пребывание на
    шуме большего уровня может привести к необратимым последствиям}                       \\
    \hline
    Реактивный двигатель         & 140   &                                                \\
                                 & 180   & \small{Необратимое полное повреждение
                                 слуховых органов}                                        \\
    Самый громкий возможный звук & 194   & \small{Интересно, почему?..}                   \\
  \end{longtable}
\end{center}

%%% Local Variables:
%%% mode: latex
%%% TeX-master: "rpz"
%%% End:
